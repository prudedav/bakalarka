 \input ctustyle
 \input glosdata
 \input opmac-bib
 \worktype [B/CZ]
 \faculty {F3}
 \department {Katedra elektromagnetického pole}
 \title {Modelování bezdrátového spojení mezi družicí a říční lodí}
 \author {David Prudek}
 \date {duben 2017}
 \abstractEN {Testing}
 \abstractCZ {Testovani}
 \declaration {Prohlasuji, ze jsem pracoval poctive.}
 \makefront
 
 \def\frac#1#2{{\begingroup#1\endgroup\over#2}} %/frac je LaTex, definujme vlastni

 \chap Motivace
 
 S narůstajícím počtem mobilních zařízení, ať se již jedná o mobilní
telefony, GPS navigace nebo jiné mobilní radiostanice a zvyšujícími se nároky
na kvalitu přenosu signálu, vzniká potřeba teoreticky modelovat šíření signálu
pro konkrétní prostředí. Pro mobilní zařízení je nejdůležitější se zabývat
modelováním šíření podél frekventovaných dopravních cest, jako jsou dálnice,
hlavní železniční tratě nebo frekventované říční trasy.
	
	Především z důvodu, že pro dálniční nebo železniční sítě je již v této době
hotovo poměrně velké množství modelů, zabývá se tato práce vícecestným
šířením v říčním korytu.

	Model v prostředí MATLAB zahrnuje několik základních situací šíření
signálu od vysílače (pro nás satelit) k přijímači (pro nás loď plující středem
koryta). Základní situace pro koryto řeky: Přímý paprsek (přímá viditelnost mezi
anténami), paprsek odražený od vodní hladiny a paprsky odražené od stěn
koryta. Posledním typem je dvojitý doraz (od stěny koryta a následně od vodní
hladiny).Simulace pak obsahuje také model mostu zjednodušený na dvě
dokonale ostré hrany, na nichž dochází k difrakci. Základní situace pro most
jsou: Difrakce na horní a spodní hraně mostu (po difrakci se signál šíří dále
přímo k přijímači), odraz vlny od vodní hladiny po difrakci na hranách mostu.

	V jednotlivých částech práce jsou prezentovány situace, které v simulaci
uvažujeme a jsou v nich zmíněny základní vztahy použité pro výpočet.
Pro každou část jsou uvedeny grafy přijatého výkonu přijímačem.
 
 
 \chap Odraz elektromagnetické vlny

K odrazu nějakého vlnění dochází obecně vždy na rozhraní dvou
prostředí s různými vlastnostmi. Pro elektromagnetickou vlnu jsou tyto
vlastnosti permitivita ($\varepsilon$) a permeabilita ($\mu$). Na těchto parametrech závisí, jak
velká část elektromagnetické vlny projde rozhraním do prostředí s jinými
parametry a jaká část se od rozhraní odrazí. Podle obr. 1 dopadá na rozhraní
dvou prostředí elektromagnetická vlna pod úhlem $\theta$ i Z Maxwellových rovnic
vyplývá, že po dopadu na rozhraní vzniknou z dopadající vlny dvě vlny nové
(odražená a procházející) a to se stejnou frekvencí jako vlna dopadající.
Vzhledem k tomu, že vlna procházející do druhého prostředí je zatlumena
ještě dalšími interakcemi s překážkami, ji do simulací nezahrnujeme (měla by
jen zanedbatelný vliv).
 
\chap Difrakce

 
 \chap Vícecestné šíření
 
Přítomnost zemské atmosféry a terénních překážek (např.: povrch země,
zástavba, vodní plochy) v blízkosti přenosové cesty má za následek interakci
elektromagnetické vlny s překážkami a vznik tzv. vícecestného šíření.
V takovýchto situacích dochází k přijetí také druhotných vln odražených od
překážek. Ve velmi komplikovaném terénu (typicky městská zástavba) dochází
k přijetí nekonečně mnoha odražených vln. V takovém případě volíme pro
simulace jen takové odražené vlny, které zásadně ovlivní výsledek (čím vícekrát
se vlna odrazí a po čím delší dráze se šíří, tím se vlna více utlumí).

Situace vícecestného šíření se dá zjednodušit pomocí tzv. paprskové
optiky. V té se využíva zjednodušení vlny na diskrétní paprsek (zanedbání šířky
Fresnelovy zóny). Toto používáme ale pouze v případě, že je vlnová délka
podstatně menší než velikosti objektů účastnících se interakce. [2]

Na straně přijímací antény se vlny přijaté z jednotlivých cest šíření
vektorové sčítají (1).

 
	$$ \overline{E}=\sum\limits_{i} \overline{E_l}\eqmark $$
 
Kde $\overline{E_l}$ je příspěvek od jednotlivých paprsků a $\overline{E}$ je výsledný vektor intenzity v místě přijímací antény [2].
\bigskip
Problematika vícecestého šíření take znamená, že má každá cesta šíření
jinou délku. Tudíž se každou cestou šíří signál po jiný časový úsek a nesená
informace je přijata pokaždé v jiném čase. Toto není ale pro tuto prácí
podstatné, týká se to spíše následného zpracování signálu.
\bigskip
V modelu neuvažujeme šířku Fresnelovy zóny, pouze spojnice bodů.
Toto zjednodušení pomocí paprskové optiky funguje ale pouze pokud považujeme všechny objekty se kterými vlna interaguje za mnohem větší než je vlnová délka signálu.
Zastínění mostem je realizováno poklesem intenzity elektrického pole přímého
paprsku při průchodu pod mostem na nulu.

\sec Šíření na přímou viditelnost

Šíření na přímou viditelnost je nejjednodušší případ, který může
v našem případě nastat. Při této situaci se šíří signál pouze po spojnici vysílače
s přijímačem. Průběh závislosti přijatého signálu na vzájemné vzdálenosti
vysílače a přijímače lze jednoduše vypočítat jen z útlumu způsobeného volným
prostorem.

$$
\label[rovprima]
E=\frac{E_0}{d_pr}e^{-jkd_pr}\eqmark $$

Kde $E_0$ je počáteční amplituda v místě vysílací antény, $k$ konstanta šíření a $d_pr$ přímá vzdálenost vysílací a přijímací antény [1], [2].
 
 
 
\medskip \clabel[primavid]{Nasimulovaný průběh signálu při šíření pouze přímého paprsku bez zastínění mostem.}
\picw=11cm \cinspic primaviditelnost.png
\caption/f Nasimulovaný průběh signálu při šíření pouze přímého paprsku bez zastínění mostem. Simulace na základě vztahu \ref[rovprima]
\medskip %obrazek zustal tam, kde ma. \midinsert ho posunul

\medskip \clabel[primavid2]{Nasimulovaný průběh signálu při šíření pouze přímého paprsku se zastíněním mostem.}
\picw=11cm \cinspic primaviditelnost_sezastinenimmostem.png
\caption/f Nasimulovaný průběh signálu při šíření pouze přímého paprsku se zastíněním mostem. Simulace na základě vztahu \ref[rovprima]
\medskip


\sec Dvoupaprskový model

Přidáním jednou odraženého paprsku dostáváme tzv. dvoupaprskový
model. Průběh přijatého signálu v závislosti na vzdálenosti je zde složitější.
Přímý paprsek je doplněn o odražený, který se šíří po delší trajektorii a při
odrazu od překážky se mění jeho fáze na opačnou. Komplexní signály
se v místě jejich přijetí anténou sčítají, čímž vzniká typické kolísání hladiny
signálu při změně vzdálenosti od vysílače. Při úplném odražení vlny
od překážky by se v místech s přesně opačnou fází signálů blížila hodnota
přijatého výkonu k nule. Situaci popisuje vztah \ref[dvou]

$$ 
\label[dvou]
E=\frac{E_0}{d_pr}e^{-jkd_pr}+\frac{E_0}{d_od}e^{-jkd_od}Re^{-j\psi}\eqmark $$

Kde $d_od$ je délka trajektorie odraženého paprsku, $Re^{-j\psi}$ je komplexní činitel odrazu, $R$ je činitel odrazu [2].

\bigskip
Se znalostí proudů, protékajících jednotlivými fázemi, pak můžeme vypočítat myšlené proudy tekoucí rovnoběžně s imaginární a reálnou osou.

 $$ 
 \eqalignno{ i_{s\alpha}&=k(i_{sa}-i_{sb}/2-i_{sc}/2) \cr
  i_{s\alpha}&=(2i_{sa}-i_{sb}-i_{sc})/3 & \eqmark \cr}
  $$
 
$$
\eqalignno{ i_{s\beta}&=k\sqrt{3}(i_{sb}-i_{sc})/2 \cr
	i_{s\beta}&=(i_{sb}-i_{sc})/\sqrt{3}  & \eqmark \cr}
$$

Pro volbu $k=2/3$.

Pro popis PMS motorů je uvažován ideálně symetrický motor se sinusoidně rozloženým vinutím. Pro takovou idealizaci uvažujeme napětí na vinutích $u_{sa}$, $u_{sb}$ a $u_{sc}$  následující:

$$
\label[rce1]
u_{sa}=R_si_{sa}+{{d}\over{dt}}\psi_{sa} \eqmark$$
$$ u_{sb}=R_si_{sb}+{{d}\over{dt}}\psi_{sb} \eqmark$$
$$ u_{sc}=R_si_{sc}+{{d}\over{dt}}\psi_{sc} \eqmark$$

kde $\psi_{sa}$,$\psi_{sb}$ a $\psi_{sc}$ jsou magnetické indukční toky vyvolané proudy odpovídajících vinutí. Vyjádření napětí vektorem v Gaussově(komplexní) rovině (Clarkova transformace) bude následující:

 $$ u_{s\alpha}=R_si_{s\alpha}+{{d}\over{dt}}\psi_{s\alpha} \eqmark$$
 $$ u_{s\beta}=R_si_{s\beta}+{{d}\over{dt}}\psi_{s\beta} \eqmark$$
 
 Přitom složky magnetického indukčního toku budou následující:
 
 $$\psi_{s\alpha} = L_{s\alpha}i_{s\alpha} + \psi_Mcos\theta_r \eqmark$$
 $$\psi_{s\beta} = L_{s\beta}i_{s\beta} + \psi_Msin\theta_r \eqmark$$
 
 kde $\theta_r$ je úhlová pozice rotoru a $\psi_M$ je magnetický indukční tok rotoru. $L_{s\alpha}$ a $L_{s\beta}$ jsou složky vzájemné indukčnosti rotor-stator.  

Úhlové zrychlení takového motoru o zátěži $T_L$ s $p$ póly připadajícími na každou fázi můžeme vyjádřit jako:

$$
{d\omega\over{dt}}=
{p\over{j}}\lbrack{3\over{2}}p(\psi_{s\alpha}i_{s\beta}-\psi_{s\beta}i_{s\alpha})-T_L\rbrack
$$

Rovnice \ref[rce1].

 




 
 
 
 
 \sec Konstrukce
 
 Text v sekci.
 
 \sec Řízení
 
 


%clabel prida referenci - muzu odkazovat \ref + prida do seznamu tabulek / obrazku
%midinsert, topinsert - chce na zacatek stranky, kdyz to nejde, prejde na zacatek dalsi
\midinsert \clabel[RPi_Modely]{Seznam modelů Raspberry Pi}
\ctable{lccccc}{ 
	\hfil Model & A & A+ & B & B+ & Bv2 \crl
		Počet pinů & 26 & 40 & 26 & 40 & 40  \cr
		RAM paměť [MB]  & 256 & 256 & 256/512 & 512 & 1024   \cr
		USB porty & 1 & 1 & 2 & 4  & 4 \cr
		RJ45 & Ne & Ne & Ano & Ano  & Ano \cr
		Slot na kartu & SDHC & MicroSD & SD & MicroSD & MicroSD   \cr
		Příkon [W] & 1.5 & 1.0 & 3.5 & 3 & 4  \cr
		Takt CPU [MHz] & 700 & 700 & 700 & 700 & 900  \cr
		Jádra CPU [MHz] & 1 & 1 & 1 & 1 & 4  \cr
		CPU Arch [W] & ARMv6 & ARMv6 & ARMv6 & ARMv6 & ARMv7-A  \cr
}
\caption/t Seznam modelů Raspberry Pi 
\endinsert


 \sec Linux a RT vylepšení
 
 \sec FPGA
 
 Text v sekci.\cite[AGL125] \cite[Libero_ug] \cite[rt-wiki]
 
 \chap Použité řešení
 
 Text v sekci.
 
 \chap Závěr
 

 



 \bibchap
 \usebib/c (simple) mybase

\app Zadání

 \sec pokyny
 
Na platformě procesorové desky Raspberry Pi implementujte systém pro řízení bezkartáčových (BLDC/PMSM) motorů.

1. Pro komunikaci procesorového systému s výkonovým hardwarem realizovaným s využitím programovatelného obvodu (FPGA) vyberte vhodný protokol a periferii.

2. Pro vybraný způsob komunikace navrhněte ovladač na straně jádra Linux a obvodový návrh ve VHDL na straně FPGA.

3. Integrujte bloky pro snímání polohy, řízení výkonových stupňů a měření proudu do FPGA návrhu.

4. S využitím navržených periferií realizujte řízení bezkartáčového motoru.

5. Vyžaduje se podrobná technická dokumentace včetně přípravy podkladů pro prezentaci včetně videozáznamu.

\app Zkratky\par \makeglos %vlozi novou prilohu



 
 
 \bye
  
 