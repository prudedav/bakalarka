 \input ctustyle
 \input glosdata
 \input opmac-bib
 \worktype [B/CZ]
 \faculty {F3}
 \department {Katedra elektromagnetického pole}
 \title {Modelování bezdrátového spojení mezi družicí a říční lodí}
 \author {David Prudek}
 \date {duben 2017}
 \abstractEN {Testing}
 \abstractCZ {Testovani}
 \declaration {Prohlasuji, ze jsem pracoval poctive.}
 \makefront
 
 \def\frac#1#2{{\begingroup#1\endgroup\over#2}} %/frac je LaTex, definujme vlastni

 \chap Motivace
 
 S narůstajícím počtem mobilních zařízení, ať se již jedná o mobilní
telefony, GPS navigace nebo jiné mobilní radiostanice a zvyšujícími se nároky
na kvalitu přenosu signálu, vzniká potřeba teoreticky modelovat šíření signálu
pro konkrétní prostředí. Pro mobilní zařízení je nejdůležitější se zabývat
modelováním šíření podél frekventovaných dopravních cest, jako jsou dálnice,
hlavní železniční tratě nebo frekventované říční trasy.
	
	Především z důvodu, že pro dálniční nebo železniční sítě je již v této době
hotovo poměrně velké množství modelů, zabývá se tato práce vícecestným
šířením v říčním korytu.

	Model v prostředí MATLAB zahrnuje několik základních situací šíření
signálu od vysílače (pro nás satelit) k přijímači (pro nás loď plující středem
koryta). Základní situace pro koryto řeky: Přímý paprsek (přímá viditelnost mezi
anténami), paprsek odražený od vodní hladiny a paprsky odražené od stěn
koryta. Posledním typem je dvojitý doraz (od stěny koryta a následně od vodní
hladiny).Simulace pak obsahuje také model mostu zjednodušený na dvě
dokonale ostré hrany, na nichž dochází k difrakci. Základní situace pro most
jsou: Difrakce na horní a spodní hraně mostu (po difrakci se signál šíří dále
přímo k přijímači), odraz vlny od vodní hladiny po difrakci na hranách mostu.

	V jednotlivých částech práce jsou prezentovány situace, které v simulaci
uvažujeme a jsou v nich zmíněny základní vztahy použité pro výpočet.
Pro každou část jsou uvedeny grafy přijatého výkonu přijímačem.
 
 
 \chap Odraz elektromagnetické vlny

K odrazu nějakého vlnění dochází obecně vždy na rozhraní dvou
prostředí s různými vlastnostmi. Pro elektromagnetickou vlnu jsou tyto
vlastnosti permitivita ($\varepsilon$) a permeabilita ($\mu$). Na těchto parametrech závisí, jak
velká část elektromagnetické vlny projde rozhraním do prostředí s jinými
parametry a jaká část se od rozhraní odrazí. Podle obr. 1 dopadá na rozhraní
dvou prostředí elektromagnetická vlna pod úhlem $\theta$ i Z Maxwellových rovnic
vyplývá, že po dopadu na rozhraní vzniknou z dopadající vlny dvě vlny nové
(odražená a procházející) a to se stejnou frekvencí jako vlna dopadající.
Vzhledem k tomu, že vlna procházející do druhého prostředí je zatlumena
ještě dalšími interakcemi s překážkami, ji do simulací nezahrnujeme (měla by
jen zanedbatelný vliv).
 
\chap Difrakce

 
\label [vicecest] \chap Vícecestné šíření
 
Přítomnost zemské atmosféry a terénních překážek (např.: povrch země,
zástavba, vodní plochy) v blízkosti přenosové cesty má za následek interakci
elektromagnetické vlny s překážkami a vznik tzv. vícecestného šíření.
V takovýchto situacích dochází k přijetí také druhotných vln odražených od
překážek. Ve velmi komplikovaném terénu (typicky městská zástavba) dochází
k přijetí nekonečně mnoha odražených vln. V takovém případě volíme pro
simulace jen takové odražené vlny, které zásadně ovlivní výsledek (čím vícekrát
se vlna odrazí a po čím delší dráze se šíří, tím se vlna více utlumí).

Situace vícecestného šíření se dá zjednodušit pomocí tzv. paprskové
optiky. V té se využíva zjednodušení vlny na diskrétní paprsek (zanedbání šířky
Fresnelovy zóny). Toto používáme ale pouze v případě, že je vlnová délka
podstatně menší než velikosti objektů účastnících se interakce. [2]

Na straně přijímací antény se vlny přijaté z jednotlivých cest šíření
vektorové sčítají (1).

 
	$$ \overline{E}=\sum\limits_{i} \overline{E_l}\eqmark $$
 
Kde $\overline{E_l}$ je příspěvek od jednotlivých paprsků a $\overline{E}$ je výsledný vektor intenzity v místě přijímací antény [2].
\bigskip
Problematika vícecestého šíření take znamená, že má každá cesta šíření
jinou délku. Tudíž se každou cestou šíří signál po jiný časový úsek a nesená
informace je přijata pokaždé v jiném čase. Toto není ale pro tuto prácí
podstatné, týká se to spíše následného zpracování signálu.
\bigskip
V modelu neuvažujeme šířku Fresnelovy zóny, pouze spojnice bodů.
Toto zjednodušení pomocí paprskové optiky funguje ale pouze pokud považujeme všechny objekty se kterými vlna interaguje za mnohem větší než je vlnová délka signálu.
Zastínění mostem je realizováno poklesem intenzity elektrického pole přímého
paprsku při průchodu pod mostem na nulu.


\label[primaviditelnost] \sec Šíření na přímou viditelnost

Šíření na přímou viditelnost je nejjednodušší případ, který může
v našem případě nastat. Při této situaci se šíří signál pouze po spojnici vysílače
s přijímačem. Průběh závislosti přijatého signálu na vzájemné vzdálenosti
vysílače a přijímače lze jednoduše vypočítat jen z útlumu způsobeného volným
prostorem.

$$
\label[rovprima]
E=\frac{E_0}{d_{pr}}e^{-jkd_{pr}}\eqmark $$

Kde $E_0$ je počáteční amplituda v místě vysílací antény, $k$ konstanta šíření a $d_{pr}$ přímá vzdálenost vysílací a přijímací antény [1], [2].
 
 
 
\medskip \clabel[primavid]{Nasimulovaný průběh signálu při šíření pouze přímého paprsku bez zastínění mostem.}
\picw=11cm \cinspic primaviditelnost.png
\caption/f Nasimulovaný průběh signálu při šíření pouze přímého paprsku bez zastínění mostem. Simulace na základě vztahu \ref[rovprima]
\medskip %obrazek zustal tam, kde ma. \midinsert ho posunul

\medskip \clabel[primavid2]{Nasimulovaný průběh signálu při šíření pouze přímého paprsku se zastíněním mostem.}
\picw=11cm \cinspic primaviditelnost_sezastinenimmostem.png
\caption/f Nasimulovaný průběh signálu při šíření pouze přímého paprsku se zastíněním mostem. Simulace na základě vztahu \ref[rovprima]
\medskip


\label[dvoupaprsk] \sec Dvoupaprskový model

Přidáním jednou odraženého paprsku dostáváme tzv. dvoupaprskový
model. Průběh přijatého signálu v závislosti na vzdálenosti je zde složitější.
Přímý paprsek je doplněn o odražený, který se šíří po delší trajektorii a při
odrazu od překážky se mění jeho fáze na opačnou. Komplexní signály
se v místě jejich přijetí anténou sčítají, čímž vzniká typické kolísání hladiny
signálu při změně vzdálenosti od vysílače. Při úplném odražení vlny
od překážky by se v místech s přesně opačnou fází signálů blížila hodnota
přijatého výkonu k nule. Situaci popisuje vztah \ref[dvou]

$$ 
\label[dvou]
E=\frac{E_0}{d_{pr}}e^{-jkd_{pr}}+\frac{E_0}{d_{od}}e^{-jkd_{od}}Re^{-j\psi}\eqmark $$

Kde $d_{od}$ je délka trajektorie odraženého paprsku, $Re^{-j\psi}$ je komplexní činitel odrazu, $R$ je činitel odrazu [2].

\medskip \clabel[dvojcest1]{Nasimulovaný průběh signálu při šíření přímého a odraženého paprsku bez zastineni mostem}
\picw=11cm \cinspic dvojcestne.png
\caption/f Nasimulovaný průběh signálu při šíření přímého a odraženého paprsku bez zastínění mostem
\medskip

\medskip \clabel[dvojcest2]{Nasimulovaný průběh signálu při šíření přímého a odraženého paprsku se zastinenim mostem}
\picw=11cm \cinspic dvojcestne_sezastinenimmostem.png
\caption/f Nasimulovaný průběh signálu při šíření přímého a odraženého paprsku se zastíněním mostem
\medskip

\sec Odraz od stěn koryta 

Přidáním dvou paprsků odražených od stěn koryta vzniká
již komplikovaná funkce, kterou nelze snadno analyticky určit. Pro simulaci
v prostředí MATLAB to znamená přidat do součtu komplexních intenzit dvě
intenzity odražených paprsků od stěn. Hlavním parametrem pro jejich výpočet
je sklon stěny koryta vůči vodorovné hladině. V simulaci zatím uvažuji
nekonečně rozlehlé roviny stěn koryta. Problémovou částí je zde také situace,
kdy se úhel odklonu stěny koryta velmi blíží nule. (Pro úhel roven nule splyne
odraz od stěny s odrazem od vody diskutovaným ve dvoupaprskovém modelu.
Pro tuto situaci je do simulace přidána podmínka nerovnosti úhlu nule.) Pro úhly
blížící se nule nastane situace, kdy by k odrazu ve skutečnosti nedošlo (bod
odrazu se nachází pod hladinou vody). V těchto případech se tedy simulace
dopouští nepřesnosti. Pro výslednou přijatou intenzitu (uvažujeme stejné úhly
sklonu rovin) dostaneme vztah \ref[odrazkoryto]. Vztahy (4), (5), (6) a (7) je pak možné využít pro výpočet geometrie koryta.



$$
\label[odrazkoryto]
E=\frac{E_0}{d_{pr}}e^{-jkd_{pr}}+\frac{E_0}{d_{od}}e^{-jkd_{od}}R_ve^{-j\psi}+2\frac{E_0}{d_{odk}}e^{-jkd_{odk}}R_ke^{-j\psi}\eqmark $$

$$
\label[dod2]
d_{od2}=\sqrt{d_2^2+(h_{S2}+h_{L2})^2}\eqmark$$

$$
\label[d2]
d_{2}=\sqrt{d_{pr}^2-(h_{S2}-h_{L2})^2}\eqmark$$

$$
\label[hs2]
h_{S2}=(a*tan \alpha+h_{S1})cos \alpha\eqmark $$

$$
\label[hl2]
h_{L2}=(b*tan \alpha+h_{L1})cos \alpha\eqmark $$

Kde veličina $a$ je vzdálenost kolmého průmětu satelitu na vodní hladinu
od břehu řeky (pro nás je to polovina šířky řeky, $a = b$ ), $b$ je vzdálenost lodě
od břehu , $h_{S2}$ je výška satelutu vůči stěně koryta , $h_{L2}$ je výška lodě vůči stěně koryta , $d_{pr}$ je délka přímého spoje satelitu a lodě , $d_2$ je vzdálenost kolmých průmětů lodě a satelitu na rovinu stěny koryta , $d_{odk}$ je délka paprsku odraženého od stěny koryta.

\medskip \clabel[koryto]{Vizualizace koryta}
\picw=11cm \cinspic bocni.png
%\caption/f Nasimulovaný průběh signálu při šíření přímého a odraženého paprsku se zastíněním mostem
\medskip

\medskip %\clabel[koryto]{Nasimulovaný průběh signálu při šíření přímého a odraženého paprsku se zastinenim mostem}
\picw=5cm \cinspic zepredu.png
\caption/f Vizualizace odrazu od stěny koryta a přímého paprsku a) boční pohled na koryto řeky, b) pohled zepředu
\medskip


\medskip \clabel[odrazykorytograf]{Simulace přijatého výkonu při odrazu od obou stěn koryta}
\picw=11cm \cinspic odrazy_dtenykoryta.png
\caption/f Simulace přijatého výkonu při odrazu od obou stěn koryta, pro simulaci byl použit vztah \ref[odrazkoryto]
\medskip

\sec Odraz od stěny koryta a následně od vodní hladiny


Tato situace je nejsložitější, kterou pro odražené paprsky uvažujeme.
Útlum odrazů od překážek je již tak velký, že se na součtu přijatého výkonu
odražený signál projeví jen minimálně. Do vztahu pro výpočet je v tomto případě
nutné přidat další faktor ovlivňující změnu fáze signálu a jeho útlum.
Vztah bude v tomto případě vypadat následovně \ref[dvojodraz].

$$ \label[dvojodraz]
E=\frac{E_0}{d_{pr}}e^{-jkd_{pr}}+\frac{E_0}{d_{od}}e^{-jkd_{od}}R_ve^{-j\psi}+2\frac{E_0}{d_{odk}}e^{-jkd_{odk}}R_ke^{-j\psi}+\frac{E_0}{d_{2od}}e^{-jkd_{2od}}R_ke^{-j\psi}R_ve^{-j\psi}\eqmark  $$

Kde $d_{2od}$ je délka dráhy dvakrát odraženého paprsku.

\medskip \clabel[odrazydvojite]{Simulace přijatého výkonu při započtení také odrazu od stěny koryta a následně od vodní hladiny}
\picw=11cm \cinspic odrazy_dvojiteodrazy.png
\caption/f Simulace přijatého výkonu při započtení také odrazu od stěny koryta a následně od vodní hladiny
\medskip

\sec Difrakce na hranách mostu



Difrakce na hranách mostu vychází z tzv. Huygensova principu, který
popisuje jev šíření elektromagnetické vlny i za překážku v cestě. Difrakce je
silně ovlivňována tvarem překážky která vlnu zastiňuje [2].


Řešení difrakce na překážce obecného tvaru je tak složitě, že se
omezujeme na řešení pouze ve zjednodušených případech. Ve všech našich
výpočtech uvažujeme pouze dokonale ostrou a dokonale absorbující překážku.
Tento případ je nejjednodušší a nazývá se difrakcí na ostrém břitu (knife edge
diffraction) [1].

\medskip \clabel[obrazekdiffrakce]{Geometrie difrakce na ostré překážce.}
\picw=11cm \cinspic diffraction_obrazek.png
\caption/f Geometrie difrakce na ostré překážce. Převzato z literatury [1].
\medskip

Míra zastínění přímého spoje je definována pomocí parametru v \ref[mirazastineni] [2].

$$
\label[mirazastineni]
 v=h\sqrt{\frac{2}{\lambda}(\frac{1}{d_1}+\frac{1}{d_2})}\eqmark $$

Kde je $\lambda$ vlnová délka a ostatní parametry jsou znázorněné na Obr. \ref[obrazekdiffrakce].

Řešení je poměrem intenzity elektrického pole $E$ v bodě přijetí signálu $P$
při zastínění překážkou a intenzity $E_p$ v bodě $P$ pro situaci bez překážky. Tento poměr vyjadřuje vztah \ref[rovdiff] [2] .

$$
\label[rovdiff]
\frac{E}{E_p}=\frac{1+j}{2}\int_v^\infty e^{-j\frac{\pi}{2}s^2} ds \eqmark $$

Tento integrál řešíme dále numericky pomocí Sinového a Kosinového
Fresnelova integrálu. Dostáváme následující zjednodušený vztah \ref[rovdiffzjedn] [4].

$$
\label[rovdiffzjedn]
 \frac{E}{E_p}=\frac{1+j}{2}[(\frac{1}{2}-C(v))-j(\frac{1}{2})-S(v)]\eqmark $$

Kde $C(v)$ je Kosínový \ref[cosinovy] a $S(v)$ Sinový \ref[sinovy] Fresnelův integrál [4].


$$ 
\label[cosinovy]
C(v)=\int_0^v cos(\frac{\pi s^2}{2})\eqmark $$

$$ 
\label[sinovy]
S(v)=\int_0^v sin(\frac{\pi s^2}{2})\eqmark $$

Po numerickém vyřešení dostáváme následující grafy závislostí $v$ na
poměru intenzit $\frac{E}{E_p}$ (Obr. \ref[zastinovanidiff]) a útlumu difrakcí $L$ (Obr. \ref[zastinovanidiffutlum]).

\medskip \clabel[zastinovanidiff]{Závislost poměru intenzit na koeficientu zastínění}
\picw=11cm \cinspic prubeh_v_zastinovani.png
\caption/f Závislost poměru intenzit na koeficientu zastínění, vychází ze vztahů \ref[rovdiffzjedn], \ref[cosinovy], \ref[sinovy]
\medskip

\medskip \clabel[zastinovanidiffutlum]{Závislost útlumu difrakcí na koeficientu zastínění}
\picw=11cm \cinspic utlum_na_v.png
\caption/f Závislost útlumu difrakcí na koeficientu zastínění, vychází ze vztahů \ref[rovdiffzjedn], \ref[cosinovy], \ref[sinovy]
\medskip

Z grafů závislostí poměru intenzit a útlumu na koeficientu zastínění
vyplývá, že pro koeficienty v hodnot blížících se k $-\infty$ se situace velmi podobá volnému prostoru [2]. Pro nás ale taková situace nenastane. Uvažujeme totiž
difrakci na hranách mostu pouze jako příspěvek k celkové intenzitě elektrického
pole v bodě přijímací antény, do přímého paprsku tedy difrakci nezahrnujeme a
při výpočtu přírůstku intenzity díky difrakci uvažujeme zastínění přímého spoje.


Pro hodnoty $v$ vyšších hodnot (obvykle větší než 2) se dá použít také
následující zjednodušený vztah pro poměr intenzit [2].

$$ 
\label[aproxdiff]
\mid\frac{E}{E_p}\mid=\frac{1}{2\pi v}\eqmark $$

\medskip \clabel[zastinovanidiffutlumsaprox]{Závislost útlumu difrakcí na koeficientu zastínění a aproximace pro $v>2$}
\picw=11cm \cinspic prubeh_v_zastinovani_saproximaci.png
\caption/f Závislost útlumu difrakcí na koeficientu zastínění a aproximace pro $v>2$, vychází ze vztahů \ref[rovdiffzjedn], \ref[cosinovy], \ref[sinovy], \ref[aproxdiff]
\medskip

V grafu vidíme, že pro hodnoty $v>2$ obě charakteristiky v podstatě splývají.

\secc Difrakce na spodní a vrchní hraně mostu

V simulaci uvažujeme difrakci na dvou hranách mostu, které jsou
dokonale ostré. Pomocí výše uvedených vztahů \ref[rovdiffzjedn], \ref[cosinovy], \ref[sinovy] jsme provedli výpočet přírůstku intenzity elektrického pole od obou případů difrakce a sečetli ho s intenzitou získanou v předchozích částech simulace.


Uvažujeme, že k difrakci dochází jen v případech, kdy se přijímač nachází
na opačné straně vůči mostu než vysílač.

\medskip \clabel[difrakcemost]{Závislost přijatého výkonu na vzdálenosti pro přidaný případ difrakce na hranách mostu}
\picw=11cm \cinspic difrakce.png
\caption/f Závislost přijatého výkonu na vzdálenosti pro přidaný případ difrakce na hranách mostu, vychází ze vztahů \ref[rovdiffzjedn], \ref[cosinovy], \ref[sinovy]
\medskip

\secc Difrakce na hranách mostu a odraz od vodní hladiny

Pro tuto situaci uvažujeme, že dojde k difrakci na hraně mostu a následně
k odrazu elektromagnetické vlny od vodní hladiny k přijímací anténě.


Hodnoty intenzity počítáme jako difrakci pro bod odrazu od vodní hladiny
a dále pomocí činitele odrazu, útlumu volným prostorem (diskutováno
v sekcích \ref[primaviditelnost], \ref[dvoupaprsk]) dopočítáme intenzitu v místě přijetí.

\medskip \clabel[difrakceodraz]{Závislost přijatého výkonu na vzdálenosti pro difrakci a následný odraz od vodní hladiny}
\picw=11cm \cinspic difrakce_odrazodvody_spravne.png
\caption/f Závislost přijatého výkonu na vzdálenosti pro difrakci a následný odraz od vodní hladiny, vychází ze vztahů \ref[rovdiffzjedn], \ref[cosinovy], \ref[sinovy]
\medskip

\sec Použité Hodnoty 

Tabulka hodnot použitých pro simulaci veškerých průněhů v kapitole \ref [vicecest].

\medskip \clabel[datasimulace]{Data použitá pro simulaci}
\ctable{lccccc}{ 
	\hfil Parametr & Hodnota & Označení & Poznámka  \crl
		Vzdálenost & 0-5000m & & proměnná vzdálenost lodi a satelitu  \cr
		Krok výpočtu & 0,8m &&krok změny vzdálenost při simulaci    \cr
		Výška lodi &1m&&výška antény nad hladinou řeky \cr
		Výška satelitu &100m& \cr
		Výška horní strany mostu &50m&   \cr
		Výška dolní strany mostu &40m&  \cr
		Vzdálenost horní hrany mostu &450m&  \cr
		Vzdálenost dolní hrany mostu &500m&  \cr
		Šířka mostu &50m&  \cr
		Činitel odrazu vody  &-0,5& \cr
		Činitel odrazu koryta &-0,8& \cr
		Frekvence &1000 MHz  \cr
		Úhel odchylky stěny koryta &15$^\circ$&  \cr
		Vzdálenost lodi od behu&  10m& \cr
		Počáteční intenzita elektrického pole& 10V/m & \cr
}
\caption/t Data použitá pro simulaci
\medskip

\chap Simulace




\bigskip
\bigskip
\bigskip
\bigskip

 
 \chap Závěrvěr
 
Cílem tohoto projektu je realizování simulace průběhu přijatého výkonu
pohybujícího se přijímače nad vodní hladinou v korytu řeky s mostem. Simulace
zahrnuje několik základních situací, které mohou pro uvažovaný objekt nastat.
Použitým prostředím pro simulace je program MATLAB. Pro zjednodušení
považujeme most za kvádr a uvažujeme u něj dvě dokonale ostré hrany na
nichž dochází k difrakci signálu. Pro případ podjezdu lodi pod mostem
uvažujeme v modelu pouze zastínění přímé viditelnosti.


Výsledky simulací odpovídají předpokladu zvětšujícího se kolísání signálu
s každou přidanou cestou šíření. Vzhledem k tomu, že v reálném prostředí by
byl satelit ve velké vzdálenosti nad zemským povrchem a ve výsledné přijaté
úrovni signálu by mělo šíření difrakcí větší podíl (kvůli komplikovanější geometrii
terénu) bylo by vhodnější použít spíše nižší frekvence v řádech jednotek GHz.
Také z grafů vyplývá, že pro vícenásobné odrazy (nebo například difrakce a
odraz) se již charakter křivky příliš nemění a cesty využívající pro šíření ještě
další odrazy již pravděpodobně budou mít na křivku zanedbatelný vliv.


V navazující práci by bylo vhodné rozšířit množství uvažovaných cest
šíření a ověřit, že pro mnohonásobné odrazy (více než trojnásobné) se již přijatý
výkon bude měnit jen zanedbatelně. Dalším rozšířením je samozřejmě přidání
dalších situací šíření signálu pro koryto a most. Také by bylo vhodné realizovat
experiment a tím ověřit nasimulovaná data
 



 \bibchap
 \usebib/c (simple) mybase

\app Zadání

 

\app Zkratky\par \makeglos %vlozi novou prilohu


h
 
 
 \bye
  
 